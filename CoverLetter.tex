\documentclass[11pt]{letter}
\usepackage{graphicx,baskervald,microtype}
\usepackage{hyperref,amsmath,xcolor}
\usepackage{pgfplots,marginnote}
\usepackage[top=0.5in,bottom=0.5in,left=1in,right=1in]{geometry}
%\newgeometry{margin=2.5cm}
\newcommand{\omamargin}[1]{\marginnote{\textbf{#1}}[7pt]}
\begin{document}
\reversemarginpar
\pagestyle{empty}
\noindent Dr. O. H. Samuli Ollila \\
\noindent Institute of Biotechnology \\
\noindent University of Helsinki, Finland \\
\noindent samuli.ollila@helsinki.fi, Tel. +358503746963 \\


Dear Editor,

Please find enclosed our manuscript entitled
{\it Inverse conformational selection in lipid-protein binding}
by A. Bacle et al. which we would like to submit for consideration to be published in the \textit{Journal of American Chemical Society}.

In the manuscript, we present the {\it Inverse conformational selection model} for lipid-protein binding.
The model illustrates how lipids in cellular membranes or in lipid nanoparticles can select
suitable conformations from wide ensemble %sampled in physiologically relevant liquid state
to bind in various biomolecules, such as proteins, drugs, RNA or viruses.
Our model is formulated based on solid state NMR experiments, molecular dynamics (MD) simulations,
and bioinformatical analysis.
First, the extensive NMR and MD simulation data collected within the NMRlipids project
(open collaboration project started by us in 2013
with 210 citations in total) revealed that 
the most common lipid types in eukaryotes and bacteria exhibit a much higher plasticity in
terms of allowed conformational states than previously anticipated.
Motivated by these results, 
we used a novel bioinformatical analysis of protein-bound lipid structures from the protein data bank (PDB),
and found that the plasticity in lipid headgroups is a characteristic feature also in lipid-protein complexes.

While lipid-protein interactions, and cell functions regulated by them,  remain
as a central topic in molecular cell biology, interactions of lipids with proteins and other biomolecules
is gaining interested also in nanobiotechnology. For example, lipid nanoparticles used as carriers
in mRNA vaccines against COVID-19 can cause allergic reactions.
Our {\it Inverse conformational selection model} can be used to understand how
these nanoparticles interact with RNA cargo and physiological environment,
thereby facilitating the design of novel vaccine or drug carrier with less side effects.
Furthermore, the {\it Inverse conformational selection model}
may apply also to other than lipid molecules, such as intrinsically disordered proteins or sugars.

We believe that the proposed conceptually novel {\it Inverse conformational selection model} 
contributes to wide scope of research fields, ranging from biophysics and biochemistry to biology and biotechnology.
Moreover, the combination of the state-of-the-art NMR experiments and MD simulations with original open collaboration
and bioinformatical approaches further increase the significance of the manuscript.
%the open access policy of {\it Nature Communications} is optimal for the publications from
%the NMRlipids Project which fosters the open collaboration and open data approaches as all the data and other material
%are publicly available throughout the project.
Therefore, we believe that our manuscript fulfills the high standards for publications in the \textit{Journal of American Chemical Society}.

We now leave it for your own consideration.

Sincerely yours,

On behalf of the authors,

Dr. O. H. Samuli Ollila

\end{document}
