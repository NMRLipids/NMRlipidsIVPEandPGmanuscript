\documentclass[11pt]{letter}
\usepackage{graphicx,baskervald,microtype}
\usepackage{hyperref,amsmath,xcolor}
\usepackage{pgfplots,marginnote}
\usepackage[top=0.5in,bottom=0.5in,left=1in,right=1in]{geometry}
%\newgeometry{margin=2.5cm}
\newcommand{\omamargin}[1]{\marginnote{\textbf{#1}}[7pt]}
\begin{document}
\reversemarginpar
\pagestyle{empty}
\noindent Dr. O. H. Samuli Ollila \\
\noindent Institute of Biotechnology \\
\noindent University of Helsinki, Finland \\
\noindent samuli.ollila@helsinki.fi, Tel. +358503746963 \\


Dear Editor,

Please find enclosed our manuscript entitled {\it Conformational plasticity of lipid headgroups in cellular membranes and protein--lipid complexes}
by A. Bacle et al. which we would like to submit for consideration to be published in \textit{Chemical Science}.

In the manuscript, we demonstrate that lipids in cellular membranes exhibit a much higher plasticity in terms of allowed conformational states than previously anticipated.
Our results update the current understanding on how lipids selectively bind to proteins
and present a significant contribution to the emerging field studying biomolecular complexes containing disordered molecules.

Lipids with different headgroups, the water facing components of cell membranes, are known to regulate cell functions
by selective binding to different proteins. It is, however, not known if the selectivity is driven by unique
lipid conformations between the lipid types that fit to binding pockets in proteins (lock and key model), or by some other interactions.
One of the main obstacles in understanding the selective binding has been the lack of
experimental methods to resolve the conformational ensembles of lipids when they are not bound to proteins.

In 2013 we set out to solve the conformational ensembles of lipid headgroups by establishing the NMRlipids open collaboration project
which combines data from molecular dynamics (MD) simulations and solid-state NMR experiments (previous publications from the project have gained 209 citations in total).
The extensive data collected within this project has now enabled us to resolve the differences in conformational ensembles between the most common lipid types in eukaryotes and bacteria. 
Our results reveal a significant structural plasticity in lipid headgroups and only minor differences of the conformational ensembles
between lipids with different headgroup chemistry.
Motivated by our findings on the model systems, we used a novel bioinformatical analysis of protein-bound lipid structures from the protein data bank (PDB), and found that the plasticity in lipid headgroups is a characteristic feature also in lipid-protein complexes.

The novel conclusions from our work are that lipid headgroups can adapt to multiple binding sites in proteins,
and the selectivity of lipid-protein binding is driven by intermolecular interactions rather than by restrictions in lipid conformations.
Furthermore, we demonstrated the power of a conceptually novel open collaboration approach that extended the realm of MD simulations to complement the experimental data in the PDB for elucidating how complex systems made up of disordered biomolecules behave.

We believe that our approach and findings are of great interest to understand the connection of lipid chemistry to protein-lipid interactions, and ultimately the role of lipid chemistry in a number of diseases. Moreover, the open access policy of {\it Chemical Science} is optimal for the publications from the NMRlipids Project which fosters the open collaboration and open data approaches as all the data and other material are publicly available throughout the project. Therefore, we believe that \textit{Chemical Science} is a highly suitable journal to publish this work.   

We now leave it for your own consideration.

Sincerely yours,

On behalf of the authors,

Dr. O. H. Samuli Ollila

\end{document}
