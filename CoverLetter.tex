\documentclass[11pt]{letter}
\usepackage{graphicx,baskervald,microtype}
\usepackage{hyperref,amsmath,xcolor}
\usepackage{pgfplots,marginnote}
\usepackage[top=0.5in,bottom=0.5in,left=1in,right=1in]{geometry}
%\newgeometry{margin=2.5cm}
\newcommand{\omamargin}[1]{\marginnote{\textbf{#1}}[7pt]}
\begin{document}
\reversemarginpar
\pagestyle{empty}
\noindent O. H. Samuli Ollila \\
\noindent Institute of Biotechnology \\
\noindent University of Helsinki, Finland \\
\noindent samuli.ollila@helsinki.fi, Tel. +358503746963 \\


Dear Editor,

Please find enclosed our manuscript entitled {\it Conformational plasticity of phospholipid headgroups in simulations and experiments}
by A. Bacle et al. which we would like to submit for consideration of publication in the \textit{ACS Central Science}.

In the manuscript, we demonstrate that lipids in cellular membranes exhibit more significant plasticity than previously anticipated.
Our results update the current understanding on how lipids selectively bind to proteins.

Lipids with different headgroups, the water facing components of cell membranes, are known to regulate cell functions
by selective binding to different proteins. It is, however, not known if the selectivity is driven by suitable
lipid conformations that fit to binding pockets in proteins (lock and key model), or by some other interactions.
One of the main obstacles in understanding the selective binding has been the lack of
experimental methods to resolve the conformational ensembles of lipids when they are not bound to proteins.

In 2013 we set out to solve the conformational ensembles of lipid headgroups by establishing the NMRlipids open collaboration project
which combines data from molecular dynamics (MD) simulations and NMR experiments.
Using the extensive data collected within this project, we have now resolved the differences in conformational ensembles
between most common lipids in eukaryotes and bacteria. 
Our results reveal significant structural plasticity in lipid headgroup and only minor differences in conformational ensembles
between lipids with different headgroup chemistry.
Bioinformatical analysis of protein-bound lipid structures from the protein data bank (PDB) revealed the plasticity in
lipid headgroups also when bound to proteins.

Novel conclusions from our work are that lipid headgroups can adapt to multiple binding sites in proteins,
and the selectivity of lipid-protein binding is driven by intermolecular interactions rather than by restrictions in lipid conformations.
Furthermore, we demonstrated the power of conceptually novel open collaboration approach
to complement the data in the PDB for elucidating how complex systems made up of disordered biomolecules behave.




Sincerely yours,

On behalf of the authors,

O. H. Samuli Ollila

\end{document}
